\chapter{Future Work}
Much remains to be done in the area of biology and medicine. Particularly medicine. The curriculum of doctors is already so overloaded, that it is difficult to insert new courses. As a consequence, doctors know little or no machine learning, they only know classical statistics. Only few people have a double curriculum.

The big paradigm shift of machine learning will, in time, be adopted in all areas of biology and medicine. Instead of the traditional scientific method:
\begin{itemize}
\item human observation
\item modelling using prior knowledge
\item experimental testing of "the" model (or one model against a previous one).
\end{itemize}
the new "big data" era, brings a new way of doing science:
\begin{itemize}
\item collect lots of data
\item automatically generate lots of models and select them computationally
\item design new experiments to test a few selected models.
\end{itemize}
This is extremely different. Current practice in biology and medicine, based on statistical testing loosely (or not at all) taking into account the problem of multiple testing, lead to many false discoveries, as notes by professor Ioannidis. Dealing with big data requires new tools and a deep understanding of modern statistics, as taught by Vapnik, Hastie, Tibshirani, and others.

Once properly understood by biologist and doctors, Machine Learning will considerably accelerate research in these areas, till then, we just hope that it remains on the top of the most researched areas in Computer Science.
