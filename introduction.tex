\chapter{Introduction to Machine Learning}
\section{Soft Computing \& Machine Learning}
\subsection{What is Soft Computing?}
Soft computing is a collection of algorithms that are employed for finding a solution for very complex problems; the ones for which more conventional methods have not yielded low cost \& time-feasible solutions. When we are speaking of low cost, we are calling for the cost of an algorithm i.e. the how much space the data structures take and how much time taken for the algorithm to run. Space and time are the two main parameters for measuring the efficiency of an algorithm. According to L.A. Zadeh ---
\begin{framed}
\begin{quote}
{\large \textit{"The guiding principle of soft computing is: Exploit the tolerance for imprecision, uncertainty, partial truth, and approximation to achieve tractability, robustness and low solution cost."}}
\end{quote}
\end{framed}
\subsection{Relation Between Machine Learning and Soft Computing}
The primary types of algorithms which are used for Soft Computing ---
\begin{itemize}
\item Fuzzy Systems
\item Neural Networks
\item Evolutionary Computation
\item Machine Learning
\item Probabilistic Reasoning.
\end{itemize}
Given the current data mining techniques employed by large institutes and organizations, they all employ Soft Computing techniques in one way or the other. Be it Image processing or facial identification or customer profiling or fraud detection or social media behaviour discovery. All these projects involve soft computing not to mention their usage in bio-informatics and genetic programming.
Machine learning is the basic connection between all the algorithms that are used for soft computing. It can be considered as an integral part of AI and soft computing as, if a machine can't learn it can neither take decisions nor apply new things or complex solutions to new problems.
\section{Machine Learning Algorithms}
Machine learning has several algorithms that are frequently used in real life problem solving. --- 
\begin{itemize}
	\item Decision tree learning.
	\item Association rule learning. 
	\item Artificial neural networks.
	\item Deep learning.
	\item Inductive logic programming.
	\item Support vector machines.
	\item Clustering.
	\item Bayesian Networks.
	\item Reinforcement learning.
	\item Representation learning.
	
\end{itemize}