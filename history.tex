\chapter{A Brief History of Machine Learning}
The very first concept of machine learning dates back to the second World War. To be precise, the Bomb\'e that was built to crack the codes of German Enigma machine used maybe the first instances of machine learning. The machine's ability to automatically apply complex mathematical calculations to big data - iteratively and quickly - is gaining momentum only in recent times.

Machine learning grew out of the quest for AI. Already in the early days of AI as an academic discipline, some researchers were interested in having machines learn from data that was constantly generated by the world. 

Machine learning, recognized as a seperate field, started to flourish in the 1990s. The field change its gola from achieving AI to tackling solvable problems of a practical nature. It shifted focus away from the symbolic approaches it had inherited from AI, and moved towards methods and models borrowed from statistics and probability theory. It has also benifitted from the increasing availabillity of digitized information, and the possibility to distribute that via the internet.
\section{Machine Learning and Other Fields}
Machine learning and data mining often employ the same methods and overlap significanty, but while machine learning focuses on prediction, based on known properties learned from training data, data mining focuses on the discovery of previously unknown properties in data. Data mining uses many machine learning methods, but with different goals; on the other hand, machine learning also employs data mining methods as "unsupervised learning" or as a preprocessing step to improve learner accuracy. Evaluated with respect to known knowledge, an uninformed (unsupervised) method will easily be outperformed by other supervised methods, while in a typical KDD (Knowledge Discovery and Data Mining) task, supervised methods cannot be used due to the unavailability of training data.
\section{Statistics Plays a Vital Role}
Machine learning and statistics are closely related fields. According to Michael I. Jordan, the ideas of machine learning, from methodological principles to theoretical tools, have had a long pre-history in statistics. He also suggested the term data science as a placeholder to call the overall field.

Leo Breiman distinguished two statistical modelling paradigms: data model and algorithmic model, wherein 'algorithmic model' means more or less the machine learning algorithms like Random forest.

Some statisticians have adopted methods from machine learning, leading to a combined field that they call statistical learning.